%\documentclass[12pt,twoside,parskip,solutions]{handout}
\documentclass[12pt,twoside,parskip]{handout}

\renewcommand{\FirstPageHeader}{\headerZGpic{\resizebox{1in}{!}{\squircle[cyan]}}}

\course{Algebra 1A}
\schoolyear{2019--20}
\block{2}
\unit{Equations}
\unitnumber{3}
\sheetnumber{7}
\title{A Sample Handout}

\singlespacing

\begin{document}
Lorem ipsum dolor sit amet, consectetuer adipiscing elit. Ut purus elit, vestibulum ut, placerat ac, adipiscing vitae, felis. Curabitur dictum gravida mauris. Nam arcu libero, nonummy eget, consectetuer id, vulputate a, magna. Donec vehicula augue eu neque. Pellentesque habitant morbi tristique senectus et netus et malesuada fames ac turpis egestas.
\begin{prob}[nospace, discuss]
	What does it mean to study math?
	Discuss with your neighbor.
\end{prob}
\begin{prob}[calculator]
	What is the largest number that you can think of?
	Use your calculator to find an even larger one!
	\solution
	Your answer may vary, but for me it was 23.
\end{prob}
\begin{prob}[space=2]
	Lorem ipsum dolor sit amet, consectetuer adipiscing elit. Ut purus elit, vestibulum ut, placerat ac, adipiscing vitae, felis.
	\solution
	Curabitur dictum gravida mauris.
\end{prob}
\newpage
\begin{prob}[columns=3]
	Solve each of the following equations for whichever variable most speaks to you.
	
	\begin{prob}
		$y=mx+b$
		\solution
		$m=\frac{y-b}{x}$
	\end{prob}
	\begin{prob}
		$y=mx+b$
		\solution
		$m=\frac{y-b}{x}$
	\end{prob}
	\begin{prob}
		$y=mx+b$
		\solution
		$m=\frac{y-b}{x}$
	\end{prob}
	These next few are harder, but you can do it.
	You just have to believe in yourself!
	\ProbsInColumns{2}
	
	\begin{prob}
		$\frac{1}{a}+a=2$
		\solution
		$a=1$
	\end{prob}
	\begin{prob}
		$\frac{1}{a}+a=2$
		\solution
		$a=1$
	\end{prob}
	\begin{prob}
		$\frac{1}{a}+a=2$
		\solution
		$a=1$
	\end{prob}
	\begin{prob}
		$\frac{1}{a}+a=2$
		\solution
		$a=1$
	\end{prob}
\end{prob}
\newpage
\begin{prob}
	For how many different integers $b$ can $x^2+bx+10$ be factored with integer coefficients?
	\solution
	4
\end{prob}

\begin{conjecture*}
	Nam arcu libero, nonummy eget, consectetuer id, vulputate a, magna. Donec vehicula augue eu neque. Pellentesque habitant morbi tristique senectus et netus et malesuada fames ac turpis egestas.
\end{conjecture*}
\begin{prob}[space=3, exciting]
	Prove the above conjecture for $n=3$.
	\solution
	You're on your own for this one!
\end{prob}
\spoilerbreak[When you finish, ask me for the next handout!]
\end{document}
